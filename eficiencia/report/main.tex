\documentclass{article}
\usepackage[left=3cm,right=3cm,top=2.5cm,bottom=2cm]{geometry} % page settings
\usepackage{amsmath} % provides many mathematical environments & tools
\usepackage{amssymb}
\usepackage{amsfonts}
\usepackage[spanish]{babel}



\usepackage{multirow}

\usepackage{algorithm}
\usepackage{algpseudocode}
\usepackage{pifont}

\usepackage[utf8]{inputenc}
\setlength{\parindent}{0mm}

\usepackage[parfill]{parskip}

% Para el código
\usepackage{listings}
\usepackage{xcolor}
\definecolor{gray}{rgb}{0.5,0.5,0.5}
\newcommand{\n}[1]{{\color{gray}#1}}
\lstset{numbers=left,numberstyle=\small\color{gray}}

% Entorno para estilo de ejercicios
\setlength{\parindent}{0pt}

\usepackage{color}   %May be necessary if you want to color links
\usepackage{hyperref}
\hypersetup{
    colorlinks=true, %set true if you want colored links
    linktoc=all,     %set to all if you want both sections and subsections linked
    linkcolor=blue,  %choose some color if you want links to stand out
}

\usepackage{graphicx}
\usepackage{subfig}

\begin{document}

\title{%
  \huge Análisis de la eficiencia de algoritmos \\[5mm]
  \Large Algorítmica\\
  \normalsize Doble Grado en Ingeniería Informática y Matemáticas\\[5cm]
}
\author{Yábir García Benchakhtir \\ yabirgb@correo.ugr.es \\[10cm]}

\date{\today}
\maketitle

\newpage
\tableofcontents
\newpage

\section{Análisis de los algoritmos}

\section{Cálculo de la eficiencia empírica}

\subsection{Procedimiento}

Para el cálculo de la eficiencia empírica se ha automatizado el
proceso. Para ellos se han hecho 2 scripts de bash y un archivo
makefile para realizar las siguientes tareas:

\begin{itemize}
\item Crear archivos ejecutables para todos los algoritmos programados en \textit{C++} con distintas opciones de optimización a saber \textit{O1}, \textit{O2} y \textit{O3}.
\item Ejecutar los distintos tests para los intervalos programados y
  almacenar los resultados en archivos de datos.
\item Crear las respectivas gráficas para cada tabla de datos obtenida usando la herramienta gnuplot.
\end{itemize}

Para los algoritmos de medición se ha elegido un rango de datos común
en el intervalo [1000, 25000] de manera que se toman 25 medidas
haciendo incrementos de 1000 en 1000 para tomar las medidas.

$$D = \{x \in [1000, 25000]: x = 1000k, k \in \mathbb{N}\}$$

Junto a este documento se encuentran las gráficas creadas y los datos
que proporciona el programa \textit{gnuplot} sobre las mediciones.

\subsection{Condiciones de las mediciones}

Para llevar a cabo las mediciones se ha utilizado un ordenador con las siguientes características:

\begin{itemize}
\item CPU: Intel Pentium G3258 (2) @ 3.200GHz
\item Memoria RAM: 7876MiB
\item Kernel:4.13.0-36-generic
\item OS: Linux Mint 18.3 Sylvia x86\_64
\end{itemize}

A la hora de realizar los tests se ha tenido la precaución de
minimizar el uso de \textit{CPU} para no interferir en las mediciones.

\section{Resultados obtenidos}

En esta sección me he concentrado en el análisis de los resultados
obtenidos con la optimazión \textit{O0} que proporciona \textit{g++}
debido al tiempo limitado del que disponemos. Los resultados se
encuentran adjuntos para el resto de optimizaciones. Al final del
documento se hará referencia a estos datos y se usaran para
ejemplificar el efecto de la optimización.

Por simplicidad del documento los datos se recogeran en las tablas que
los comparan diferenciando claramante a que función pertenece cada
dato. Al final del documento se acompañan las medidas para las
optimizaciones de orden 1 y orden 2 para las que se han hecho medidas.

\subsection{Algoritmo de ordenación Burbuja}


\begin{figure}[H]%
    \centering
    \subfloat[Nube de puntos]{\includegraphics[width=0.5\textwidth]{../plots/burbuja_O0_points.png}}%
    \subfloat[Función continua]{\includegraphics[width=0.5\textwidth]{../plots/burbuja_O0_lines.png}}%
    \caption{Resultados experimentales representados mediante una nube de puntos y la linea que los une}%
    \centering
    \subfloat{\includegraphics[width=0.6\textwidth]{../plots/burbuja_O0_fit.png}}%
    \caption{Ajuste para: ordenación usando el algoritmo de burbuja}%
\end{figure}

\begin{verbatim}
Final set of parameters            Asymptotic Standard Error
=======================            ==========================

a               = 3.11154e-09      +/- 1.128e-11    (0.3624%)
b               = -5.17483e-06     +/- 3.02e-07     (5.837%)
c               = 0.00563209       +/- 0.001704     (30.26%)


\end{verbatim}

\subsection{Algoritmo de ordenación por insercción}



\begin{figure}[H]%
    \centering
    \subfloat[Nube de puntos]{\includegraphics[width=0.5\textwidth]{../plots/insercion_O0_points.png}}%
    \subfloat[Función continua]{\includegraphics[width=0.5\textwidth]{../plots/insercion_O0_lines.png}}%
    \caption{Resultados experimentales representados mediante una nube de puntos y la linea que los une}%
    \centering
    \subfloat{\includegraphics[width=0.6\textwidth]{../plots/insercion_O0_fit.png}}%
    \caption{Ajuste para: ordenación usando el algoritmo de inserción}%
\end{figure}

\begin{verbatim}
Final set of parameters            Asymptotic Standard Error
=======================            ==========================

a               = 1.08271e-09      +/- 6.279e-12    (0.5799%)
b               = -7.7687e-08      +/- 1.682e-07    (216.5%)
c               = 0.000123925      +/- 0.0009489    (765.7%)


\end{verbatim}


\subsection{Algoritmo de ordenación por selección}


\begin{figure}[H]%
    \centering
    \subfloat[Nube de puntos]{\includegraphics[width=0.5\textwidth]{../plots/seleccion_O0_points.png}}%
    \subfloat[Función continua]{\includegraphics[width=0.5\textwidth]{../plots/seleccion_O0_lines.png}}%
    \caption{Resultados experimentales representados mediante una nube de puntos y la linea que los une}%
    \centering
    \subfloat{\includegraphics[width=0.6\textwidth]{../plots/seleccion_O0_fit.png}}%
    \caption{Ajuste para: ordenación usando el algoritmo de selección}%
\end{figure}

\begin{verbatim}
Final set of parameters            Asymptotic Standard Error
=======================            ==========================

a               = 1.30761e-09      +/- 1.868e-12    (0.1429%)
b               = -2.41748e-07     +/- 5.005e-08    (20.7%)
c               = 0.000572432      +/- 0.0002824    (49.33%)


\end{verbatim}


\subsection{Comparativa de los algoritmos cuadráticos de ordenación}

\subsection{Algoritmo de ordenación mergesort}

\begin{figure}[H]%
    \centering
    \subfloat[Nube de puntos]{\includegraphics[width=0.5\textwidth]{../plots/seleccion_O0_points.png}}%
    \subfloat[Función continua]{\includegraphics[width=0.5\textwidth]{../plots/seleccion_O0_lines.png}}%
    \caption{Resultados experimentales representados mediante una nube de puntos y la linea que los une}%
    \centering
    \subfloat{\includegraphics[width=0.6\textwidth]{../plots/seleccion_O0_fit.png}}%
    \caption{Ajuste para: ordenación usando el algoritmo de selección}%
\end{figure}

\begin{verbatim}
Final set of parameters            Asymptotic Standard Error
=======================            ==========================

a               = 1.30761e-09      +/- 1.868e-12    (0.1429%)
b               = -2.41748e-07     +/- 5.005e-08    (20.7%)
c               = 0.000572432      +/- 0.0002824    (49.33%)


\end{verbatim}

\subsection{Algoritmo de ordenación quicksort}

\begin{figure}[H]%
    \centering
    \subfloat[Nube de puntos]{\includegraphics[width=0.5\textwidth]{../plots/quicksort_O0_points.png}}%
    \subfloat[Función continua]{\includegraphics[width=0.5\textwidth]{../plots/quicksort_O0_lines.png}}%
    \caption{Resultados experimentales representados mediante una nube de puntos y la linea que los une}%
    \centering
    \subfloat{\includegraphics[width=0.6\textwidth]{../plots/quicksort_O0_fit.png}}%
    \caption{Ajuste para: ordenación usando el algoritmo quicksort}%
\end{figure}

\begin{verbatim}
Final set of parameters            Asymptotic Standard Error
=======================            ==========================

a               = 0.000967764      +/- 0.0001992    (20.58%)
b               = 1.00236          +/- 832.3        (8.303e+04%)
c               = -0.00746142      +/- 0.00195      (26.13%)


\end{verbatim}

\subsection{Algoritmo de ordenación heapsort}

\begin{figure}[H]%
    \centering
    \subfloat[Nube de puntos]{\includegraphics[width=0.5\textwidth]{../plots/heapsort_O0_points.png}}%
    \subfloat[Función continua]{\includegraphics[width=0.5\textwidth]{../plots/heapsort_O0_lines.png}}%
    \caption{Resultados experimentales representados mediante una nube de puntos y la linea que los une}%
    \centering
    \subfloat{\includegraphics[width=0.6\textwidth]{../plots/heapsort_O0_fit.png}}%
    \caption{Ajuste para: ordenación usando el algoritmo heapsort}%
\end{figure}

\begin{verbatim}
Final set of parameters            Asymptotic Standard Error
=======================            ==========================

a               = 0.00132491       +/- 0.0002776    (20.95%)
b               = 1.00436          +/- 847.2        (8.436e+04%)
c               = -0.0102003       +/- 0.002718     (26.64%)


\end{verbatim}

\subsection{Algoritmo floyd}

\begin{figure}[H]%
    \centering
    \subfloat[Nube de puntos]{\includegraphics[width=0.5\textwidth]{../plots/floyd_O0_points.png}}%
    \subfloat[Función continua]{\includegraphics[width=0.5\textwidth]{../plots/floyd_O0_lines.png}}%
    \caption{Resultados experimentales representados mediante una nube de puntos y la linea que los une}%
    \centering
    \subfloat{\includegraphics[width=0.6\textwidth]{../plots/floyd_O0_fit.png}}%
    \caption{Ajuste para: algoritmo para calculo de costo floyd}%
\end{figure}

\begin{verbatim}

\end{verbatim}

\subsection{Algoritmo de las torres de Hanoi}

\begin{figure}[H]%
    \centering
    \subfloat[Nube de puntos]{\includegraphics[width=0.5\textwidth]{../plots/hanoi_O0_points.png}}%
    \subfloat[Función continua]{\includegraphics[width=0.5\textwidth]{../plots/hanoi_O0_lines.png}}%
    \caption{Resultados experimentales representados mediante una nube de puntos y la linea que los une}%
    \centering
    \subfloat{\includegraphics[width=0.6\textwidth]{../plots/hanoi_O0_fit.png}}%
    \caption{Ajuste para: calculo de los movimientos en las torres de hanoi}%
\end{figure}

\begin{verbatim}

\end{verbatim}


\section{Analisis de los resultados}

\subsection{Comparativa de los algoritmos $n^2$ de ordenación}

\begin{figure}[H]
  \centering   
      \subfloat {%

        \input{tables/ordenacionC0}
        
      }
      
      \subfloat{%
        \includegraphics[clip,width=0.7\columnwidth]{../plots/cuadraticas_O0.png}%
      }



\caption{Comaparación entre los distintos algoritmos de ordenación cuadráticos}
\end{figure}


\subsection{Comparativa de los algoritmos $nlogn$ de ordenación}

\begin{figure}[H]
  \centering   
      \subfloat {%

        \input{tables/ordenacionL0}
        
      }
      
      \subfloat{%
        \includegraphics[clip,width=0.7\columnwidth]{../plots/logaritmicos_O0.png}%
      }
\caption{Comaparación entre los distintos algoritmos de ordenación $nlogn$}
\end{figure}

\end{document}