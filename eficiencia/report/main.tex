\documentclass{article}
\usepackage[left=3cm,right=3cm,top=2.5cm,bottom=2cm]{geometry} % page settings
\usepackage{amsmath} % provides many mathematical environments & tools
\usepackage{amssymb}
\usepackage{amsfonts}
\usepackage[spanish]{babel}



\usepackage{multirow}

\usepackage{algorithm}
\usepackage{algpseudocode}
\usepackage{pifont}

\usepackage[utf8]{inputenc}
\setlength{\parindent}{0mm}

\usepackage[parfill]{parskip}

% Para el código
\usepackage{listings}
\usepackage{xcolor}
\definecolor{gray}{rgb}{0.5,0.5,0.5}
\newcommand{\n}[1]{{\color{gray}#1}}
\lstset{numbers=left,numberstyle=\small\color{gray}}

% Entorno para estilo de ejercicios
\setlength{\parindent}{0pt}

\usepackage{color}   %May be necessary if you want to color links
\usepackage{hyperref}
\hypersetup{
    colorlinks=true, %set true if you want colored links
    linktoc=all,     %set to all if you want both sections and subsections linked
    linkcolor=blue,  %choose some color if you want links to stand out
}

\usepackage{graphicx}
\usepackage{subfig}

\usepackage{listings}
\lstset{language=C++,
                basicstyle=\ttfamily,
                keywordstyle=\color{blue}\ttfamily,
                stringstyle=\color{red}\ttfamily,
                commentstyle=\color{green}\ttfamily,
                morecomment=[l][\color{magenta}]{\#}
}

\begin{document}

\title{%
  \huge Análisis de la eficiencia de algoritmos \\[5mm]
  \Large Algorítmica\\
  \normalsize Doble Grado en Ingeniería Informática y Matemáticas\\[5cm]
}
\author{Yábir García Benchakhtir \\ yabirgb@correo.ugr.es \\[10cm]}

\date{\today}
\maketitle

\newpage
\tableofcontents
\newpage

\section{Análisis de los algoritmos}

Queda por realizar un análisis teórico de los algoritmos de ordenación
pero por falta de tiempo lo entregaré en la siguiente fecha de
entrega.

\subsection{Algoritmo burbuja}

Este algoritmo funciona como `las burbujas en un refresco`. Lo que
hace es comparar de izquierda a derecha del vector cada elemento con
los adyacentes de modo que los más grandes se desplazan a la derecha.

Como se explica en el guión la eficiencia de este algoritmo es:

$$ \frac{a}{2}n^2 - \frac{3a}{2}n + a $$

\subsection{Algoritmo de ordenación por insercción}

Busca en un vector de elementos el menor y lo coloca a la izquierda,
después repetimos esta operación para los $n-1$ elementos restantes.

\subsection{Algoritmo de ordenación por selección}

Partimos de una lista de elementos ordenados con un solo elemento y de
otra de elementos desordenados con $n-1$ elementos. En cada paso
elegimos un elemento de la lista de elementos desordenados y lo
insertamos de manera ordenada en la lista de elementos ordenados.

\begin{lstlisting}
static void seleccion_lims(int T[], int inicial, int final)
{
  int i, j, indice_menor;
  int menor, aux;
  for (i = inicial; i < final - 1; i++) {
    indice_menor = i;
    menor = T[i];
    for (j = i; j < final; j++)
      if (T[j] < menor) {
	indice_menor = j;
	menor = T[j];
      }
    aux = T[i];
    T[i] = T[indice_menor];
    T[indice_menor] = aux;
  };
}

\end{lstlisting}

% Su eficiencia teorica es la siguiente:

% \begin{align}
%   \sum_{i=1}^{n-2}\left[c_1+\sum_{j=1}^{n-1}c_2\right] & = \sum_{i=1}^{n-2}\left[c_1+c_2\sum_{j=1}^{n-1}1\right] \\
%                                                        & = \sum_{i=1}^{n-2}\left[c_1+c_2\frac{(n-1)n}{2}\right] \\
%                                                        & = c_2\frac{(n-1)n}{2} + c_1 \sum_{i=1}^{n-2}1 \\
%                                                        & = c_2\frac{(n-1)n}{2} + c_1 \frac{(n^2-3n+2)}{2} \\
%                                                        & = n^2(c_2+c_1) - n(\frac{c_2}{2} + \frac{3c_1}{2}) + c_1
% \end{align}

\subsection{Algoritmo de ordenación mergesort}

Creamos listas de un solo elementos y las vamos combinando en
distintos pasos de manera ordenada. Estamos haciendo algoritmo que
consiste en dividir la complejidad en tareas más pequeñas y después
juntar el resultado ya más sencillo.

La complejidad de este algoritmo es $O(n\cdot log(n)$ más concretamente:

$$T(n) = c_1n + c_2n\cdot log(n)$$


\subsection{Algoritmo de ordenación quicksort}

En este algoritmo tomamos un punto que hará de pivote. Movemos los
elementos de manera que a la derecha del pivote queden los que son
mayores que él y la izquierda los que son menores. Una vez hecho esto
repetimos el proceso tomando pivotes en los conjuntos de los elementos
mayores y menores.

\subsection{Algoritmo de ordenación heapsort}

Disponemos todos los elementos en un \textit{heap}. Buscamos el mayor
de los elementos y lo quitamos insertandolo al final. Funciona de
manera parecida a la ordenación burbuja.

\subsection{Algoritmo floyd}

Este algoritmo realiza todas las sumas posibles y devuelve la mayor de
ellas.

\begin{lstlisting}
void Floyd(int **M, int dim)
{
	for (int k = 0; k < dim; k++)
	  for (int i = 0; i < dim;i++)
	    for (int j = 0; j < dim;j++)
	      {
		int sum = M[i][k] + M[k][j];    	
	    	M[i][j] = (M[i][j] > sum) ? sum : M[i][j];
	      }
}
\end{lstlisting}

Este algoritmo es de complejidad $O(n^3)$, más concretamente:

$$\sum_{k=0}^{n}{\sum_{i=0}^{n}{\sum_{j=0}^n{a}}} = an^3$$

\subsection{Algoritmo de las torres de Hanoi}

\begin{lstlisting}
void hanoi (int M, int i, int j)
{
  if (M > 0)
    {
      hanoi(M-1, i, 6-i-j);
      //cout << i << " -> " << j << endl;
      hanoi (M-1, 6-i-j, j);
  }
}
\end{lstlisting}

En este algoritmo aplicamos la siguiente recurrencia:

$$h(n) = 2h(n-1)$$

que se corresponde con la ecuación en recurrencias homogenea:

$$x_n = 2x_{n-1}$$

Cuya solución es a partir de un termino incial $x_0$:

$$x_n = 2^nx_0$$

\section{Cálculo de la eficiencia empírica}

\subsection{Procedimiento}

Para el cálculo de la eficiencia empírica se ha automatizado el
proceso. Para ellos se han hecho 2 scripts de bash y un archivo
makefile para realizar las siguientes tareas:

\begin{itemize}
\item Crear archivos ejecutables para todos los algoritmos programados
  en \textit{C++} con distintas opciones de optimización a saber
  \textit{O1}, \textit{O2} y \textit{O3}.
\item Ejecutar los distintos tests para los intervalos programados y
  almacenar los resultados en archivos de datos.
\item Crear las respectivas gráficas para cada tabla de datos obtenida usando la herramienta gnuplot.
\end{itemize}

Para los algoritmos de medición se ha elegido un rango de datos común
en el intervalo [1000, 25000] de manera que se toman 25 medidas
haciendo incrementos de 1000 en 1000 para tomar las medidas.

$$D = \{x \in [1000, 25000]: x = 1000k, k \in \mathbb{N}\}$$

Junto a este documento se encuentran las gráficas creadas y los datos
que proporciona el programa \textit{gnuplot} sobre las mediciones.

A la hora de hacer un ajuste de mínimos cuadrados se han usado las
siguientes funciones de ajuste:

\begin{table}[H]
  \centering
  \begin{tabular}{|c|c|}
    \hline
    Complejidad (\textit{Big O}) & Función de ajuste \\ \hline
    $O(n^2)$ & $f(x) = ax^2 + bx + c$ \\ \hline
    $O(n^3)$ & $f(x) = ax^3 + bx^2 + cx + d$ \\ \hline
    $O(nlog(n))$ & $f(x) = a\cdot log(x+b) + c$ \\ \hline
    $O(2^n)$ & $f(x) = a2^x$ \\ \hline


  \end{tabular}
  \caption{Funciones de ajuste}
\end{table}

\subsection{Condiciones de las mediciones}

Para llevar a cabo las mediciones se ha utilizado un ordenador con las
siguientes características:

\begin{itemize}
\item CPU: Intel Pentium G3258 (2) @ 3.200GHz
\item Memoria RAM: 7876MiB
\item Kernel: 4.13.0-36-generic
\item OS: Linux Mint 18.3 Sylvia x86\_64
\end{itemize}

A la hora de realizar los tests se ha tenido la precaución de
minimizar el uso de \textit{CPU} para no interferir en las mediciones.

\section{Resultados obtenidos}

En esta sección de la practica me concentro en mostrar los resultados
obtenidos tras haber completado el proceso de pruebas de los distintos
algoritmos. En este caso muestro los datos con una eficiencia al
compilar \textit{O0}.

\subsection{Algoritmo de ordenación burbuja}


\begin{figure}[H]%
    \centering
    \subfloat[Nube de puntos]{\includegraphics[width=0.5\textwidth]{../plots/burbuja_O0_points.png}}%
    \subfloat[Función continua]{\includegraphics[width=0.5\textwidth]{../plots/burbuja_O0_lines.png}}%
    \caption{Resultados experimentales representados mediante una nube de puntos y la linea que los une}%
    \centering
    \subfloat{\includegraphics[width=0.6\textwidth]{../plots/burbuja_O0_fit.png}}%
    \caption{Ajuste para: ordenación usando el algoritmo de burbuja}%
\end{figure}

\begin{verbatim}
Final set of parameters            Asymptotic Standard Error
=======================            ==========================

a               = 3.11154e-09      +/- 1.128e-11    (0.3624%)
b               = -5.17483e-06     +/- 3.02e-07     (5.837%)
c               = 0.00563209       +/- 0.001704     (30.26%)


\end{verbatim}

\subsection{Algoritmo de ordenación por inserción}



\begin{figure}[H]%
    \centering
    \subfloat[Nube de puntos]{\includegraphics[width=0.5\textwidth]{../plots/insercion_O0_points.png}}%
    \subfloat[Función continua]{\includegraphics[width=0.5\textwidth]{../plots/insercion_O0_lines.png}}%
    \caption{Resultados experimentales representados mediante una nube de puntos y la linea que los une}%
    \centering
    \subfloat{\includegraphics[width=0.6\textwidth]{../plots/insercion_O0_fit.png}}%
    \caption{Ajuste para: ordenación usando el algoritmo de inserción}%
\end{figure}

\begin{verbatim}
Final set of parameters            Asymptotic Standard Error
=======================            ==========================

a               = 1.08271e-09      +/- 6.279e-12    (0.5799%)
b               = -7.7687e-08      +/- 1.682e-07    (216.5%)
c               = 0.000123925      +/- 0.0009489    (765.7%)


\end{verbatim}


\subsection{Algoritmo de ordenación por selección}


\begin{figure}[H]%
    \centering
    \subfloat[Nube de puntos]{\includegraphics[width=0.5\textwidth]{../plots/seleccion_O0_points.png}}%
    \subfloat[Función continua]{\includegraphics[width=0.5\textwidth]{../plots/seleccion_O0_lines.png}}%
    \caption{Resultados experimentales representados mediante una nube de puntos y la linea que los une}%
    \centering
    \subfloat{\includegraphics[width=0.6\textwidth]{../plots/seleccion_O0_fit.png}}%
    \caption{Ajuste para: ordenación usando el algoritmo de selección}%
\end{figure}

\begin{verbatim}
Final set of parameters            Asymptotic Standard Error
=======================            ==========================

a               = 1.30761e-09      +/- 1.868e-12    (0.1429%)
b               = -2.41748e-07     +/- 5.005e-08    (20.7%)
c               = 0.000572432      +/- 0.0002824    (49.33%)


\end{verbatim}


\subsection{Comparativa de los algoritmos cuadráticos de ordenación}

\subsection{Algoritmo de ordenación mergesort}

\begin{figure}[H]%
    \centering
    \subfloat[Nube de puntos]{\includegraphics[width=0.5\textwidth]{../plots/mergesort_O0_points.png}}%
    \subfloat[Función continua]{\includegraphics[width=0.5\textwidth]{../plots/mergesort_O0_lines.png}}%
    \caption{Resultados experimentales representados mediante una nube de puntos y la linea que los une}%
    \centering
    \subfloat{\includegraphics[width=0.6\textwidth]{../plots/mergesort_O0_fit.png}}%
    \caption{Ajuste para: ordenación usando el algoritmo mergesort}%
\end{figure}

\begin{verbatim}
Final set of parameters            Asymptotic Standard Error
=======================            ==========================

a               = 0.00162555       +/- 0.0003876    (23.85%)
b               = 1.00697          +/- 964.1        (9.575e+04%)
c               = -0.0126195       +/- 0.003794     (30.07%)


\end{verbatim}

\subsection{Algoritmo de ordenación quicksort}

\begin{figure}[H]%
    \centering
    \subfloat[Nube de puntos]{\includegraphics[width=0.5\textwidth]{../plots/quicksort_O0_points.png}}%
    \subfloat[Función continua]{\includegraphics[width=0.5\textwidth]{../plots/quicksort_O0_lines.png}}%
    \caption{Resultados experimentales representados mediante una nube de puntos y la linea que los une}%
    \centering
    \subfloat{\includegraphics[width=0.6\textwidth]{../plots/quicksort_O0_fit.png}}%
    \caption{Ajuste para: ordenación usando el algoritmo quicksort}%
\end{figure}

\begin{verbatim}
Final set of parameters            Asymptotic Standard Error
=======================            ==========================

a               = 0.000967764      +/- 0.0001992    (20.58%)
b               = 1.00236          +/- 832.3        (8.303e+04%)
c               = -0.00746142      +/- 0.00195      (26.13%)


\end{verbatim}

\subsection{Algoritmo de ordenación heapsort}

\begin{figure}[H]%
    \centering
    \subfloat[Nube de puntos]{\includegraphics[width=0.5\textwidth]{../plots/heapsort_O0_points.png}}%
    \subfloat[Función continua]{\includegraphics[width=0.5\textwidth]{../plots/heapsort_O0_lines.png}}%
    \caption{Resultados experimentales representados mediante una nube de puntos y la linea que los une}%
    \centering
    \subfloat{\includegraphics[width=0.6\textwidth]{../plots/heapsort_O0_fit.png}}%
    \caption{Ajuste para: ordenación usando el algoritmo heapsort}%
\end{figure}

\begin{verbatim}
Final set of parameters            Asymptotic Standard Error
=======================            ==========================

a               = 0.00132491       +/- 0.0002776    (20.95%)
b               = 1.00436          +/- 847.2        (8.436e+04%)
c               = -0.0102003       +/- 0.002718     (26.64%)


\end{verbatim}

\subsection{Algoritmo floyd}

\begin{figure}[H]%
    \centering
    \subfloat[Nube de puntos]{\includegraphics[width=0.5\textwidth]{../plots/floyd_O0_points.png}}%
    \subfloat[Función continua]{\includegraphics[width=0.5\textwidth]{../plots/floyd_O0_lines.png}}%
    \caption{Resultados experimentales representados mediante una nube de puntos y la linea que los une}%
    \centering
    \subfloat{\includegraphics[width=0.6\textwidth]{../plots/floyd_O0_fit.png}}%
    \caption{Ajuste para: algoritmo para calculo de costo floyd}%
\end{figure}

\begin{verbatim}

\end{verbatim}

\subsection{Algoritmo de las torres de Hanoi}

\begin{figure}[H]%
    \centering
    \subfloat[Nube de puntos]{\includegraphics[width=0.5\textwidth]{../plots/hanoi_O0_points.png}}%
    \subfloat[Función continua]{\includegraphics[width=0.5\textwidth]{../plots/hanoi_O0_lines.png}}%
    \caption{Resultados experimentales representados mediante una nube de puntos y la linea que los une}%
    \centering
    \subfloat{\includegraphics[width=0.6\textwidth]{../plots/hanoi_O0_fit.png}}%
    \caption{Ajuste para: calculo de los movimientos en las torres de hanoi}%
\end{figure}

\begin{verbatim}

\end{verbatim}


\section{Analisis de los resultados}

\subsection{Comparativa de los algoritmos $O(n^2)$ de ordenación}

\begin{figure}[H]
  \centering   
      \subfloat {%

        \input{tables/ordenacionC0}
        
      }
      
      \subfloat{%
        \includegraphics[clip,width=0.7\columnwidth]{../plots/cuadraticas_O0.png}%
      }



\caption{Comaparación entre los distintos algoritmos de ordenación cuadráticos}
\end{figure}


\subsection{Comparativa de los algoritmos $nlog(n)$ de ordenación}

\begin{figure}[H]
  \centering   
      \subfloat {%

        \input{tables/ordenacionL0}
        
      }
      
      \subfloat{%
        \includegraphics[clip,width=0.7\columnwidth]{../plots/logaritmicos_O0.png}%
      }
      \caption{Comaparación entre los distintos algoritmos de ordenación $nlogn$}
\end{figure}

\subsection{Comparativa de los algoritmos de ordenación}

\begin{figure}[H]
  \centering   
      \subfloat {%
        \begin{tabular}[H]{|lllllll|}
\hline
\textbf{Tamaño} & \textbf{Burbuja} & \textbf{Inserción} & \textbf{Selección} & \textbf{Mergesort} & \textbf{Quicksort} & \textbf{Heapesort} \\ \hline
0 & 0.004246 & 0.00112 & 0.00135595 & 0.000117755 & 8.7e-05 & 0.000124 \\ \hline
1000 & 0.008201 & 0.004292 & 0.00524505 & 0.000265505 & 0.000187 & 0.000268 \\ \hline
2000 & 0.019307 & 0.009728 & 0.011705 & 0.000509884 & 0.000294 & 0.000422 \\ \hline
3000 & 0.034225 & 0.017294 & 0.020711 & 0.000580912 & 0.000394 & 0.000585 \\ \hline
4000 & 0.056845 & 0.026957 & 0.03227 & 0.000807877 & 0.000534 & 0.000748 \\ \hline
5000 & 0.088076 & 0.038868 & 0.046363 & 0.00107341 & 0.000616 & 0.00092 \\ \hline
6000 & 0.120856 & 0.052248 & 0.062962 & 0.00103389 & 0.000727 & 0.001085 \\ \hline
7000 & 0.161862 & 0.068156 & 0.082298 & 0.00124508 & 0.000843 & 0.001164 \\ \hline
8000 & 0.210286 & 0.086587 & 0.104299 & 0.00147816 & 0.00097 & 0.001303 \\ \hline
9000 & 0.264146 & 0.107304 & 0.128745 & 0.00174622 & 0.001065 & 0.00146 \\ \hline
10000 & 0.323677 & 0.130252 & 0.156038 & 0.002063 & 0.001219 & 0.001627 \\ \hline
11000 & 0.394902 & 0.15566 & 0.185971 & 0.002352 & 0.001415 & 0.001808 \\ \hline
12000 & 0.463354 & 0.182503 & 0.218305 & 0.002076 & 0.001446 & 0.001986 \\ \hline
13000 & 0.541129 & 0.212983 & 0.253628 & 0.002313 & 0.001549 & 0.00212 \\ \hline
14000 & 0.628463 & 0.243724 & 0.291057 & 0.002485 & 0.001683 & 0.002294 \\ \hline
15000 & 0.718559 & 0.2767 & 0.331193 & 0.002708 & 0.001781 & 0.002464 \\ \hline
16000 & 0.814468 & 0.309732 & 0.374365 & 0.003018 & 0.001935 & 0.002641 \\ \hline
17000 & 0.919391 & 0.346489 & 0.419854 & 0.00331 & 0.002054 & 0.002815 \\ \hline
18000 & 1.03422 & 0.387793 & 0.469411 & 0.003453 & 0.002157 & 0.00305 \\ \hline
19000 & 1.14783 & 0.432006 & 0.518467 & 0.00379 & 0.002295 & 0.003175 \\ \hline
20000 & 1.27545 & 0.479765 & 0.571629 & 0.004064 & 0.002443 & 0.003364 \\ \hline
21000 & 1.39817 & 0.524347 & 0.62718 & 0.004262 & 0.002554 & 0.003576 \\ \hline
22000 & 1.53151 & 0.569582 & 0.68739 & 0.004612 & 0.002758 & 0.003686 \\ \hline
23000 & 1.67702 & 0.620089 & 0.747746 & 0.004959 & 0.002788 & 0.003889 \\ \hline
24000 & 1.81403 & 0.675667 & 0.81211 & 0.005205 & 0.002935 & 0.004078 \\ \hline

\end{tabular}
      }
      
      \subfloat{%
        \includegraphics[clip,width=0.7\columnwidth]{../plots/todos_ordenacion_O0.png}%
      }
      \caption{Comaparación entre los distintos algoritmos de ordenación $nlogn$}
\end{figure}

\newpage

De los resultados obtenidos podemos ver de forma clara como el
algoritmo de burbuja es el que desempeña la tarea de forma
considerablemente más lenta.

Entre los algoritmos de selección e insercción no encontramos mucha
diferencia aunque son notoriamente más eficientes que los de orden
cuadrático como se observa en la comparativa.

Finalmente los algoritmos de orden $O(n\cdot log(n))$ son mucho más
eficientes que el resto aunque podemos observar como el algoritmo
quicksort ha sido el que ha obtenido mejores resultados pese a ser del
mismo orden de eficiencia.

En el algoritmo de hanoi podemos observar como al ser $O(2^n)$ el
tiempo que tarda en ejecutarse crece rapidamente a partir de un punto
lo cual nos permite realizar únicamente pocas iteraciones.

La suma de los cuadrados de los residuos de los ajustes realizados ha
sido la siguiente:

\begin{table}[H]
  \centering
  \begin{tabular}{|ll|}

    \hline
    Algoritmo & Suma residual \\ \hline
    Burbuja & 9.84713e-05 \\ \hline
    Inserción & 7.19712e-05\\ \hline
    Selección & 8.779e-06\\ \hline
    Mergesort & 1.54282e-05\\ \hline
    Heapsort & 7.24527e-06\\ \hline
    Quicksort & 2.17388e-06\\ \hline
    Floyd & 0.000654703\\ \hline
    Hanoi & 1.17475e-05\\ \hline
    
  \end{tabular}
  \caption{Ajuste los resultados}
\end{table}

Lo cual nos indica que los resultados que hemos obtenido realemte se ajustan bien a lo esperado según la notación \textit{Big  O}

Tambíen he realizado medidas con otros ordenes de eficiciencia. En
este caso expongo los resultados con la eficiencia $-O2$. Como se
puede apreciar en las gráficas, los resultados son mucho mejores y
destacan el caso de la ordenación por burbuja donde se consigue que,
aun siendo cuadrática, el tiempo de ejecución se reduzca más de la
mitad y el caso del algoritmo de Hanoi, donde pasamos de casi 14
segundos en los últimos valores a tan solo 5.

\begin{figure}[H]
  \centering
  \subfloat{\includegraphics[width=0.6\textwidth]{../plots/burbuja_comparativa.png}}
\end{figure}

\begin{figure}[H]
  \centering
  \subfloat{\includegraphics[width=0.6\textwidth]{../plots/seleccion_comparativa.png}}
\end{figure}

\begin{figure}[H]
  \centering
  \subfloat{\includegraphics[width=0.6\textwidth]{../plots/insercion_comparativa.png}}
\end{figure}

\begin{figure}[H]
  \centering
  \subfloat{\includegraphics[width=0.6\textwidth]{../plots/mergesort_comparativa.png}}
\end{figure}

\begin{figure}[H]
  \centering
  \subfloat{\includegraphics[width=0.6\textwidth]{../plots/quicksort_comparativa.png}}
\end{figure}

\begin{figure}[H]
  \centering
  \subfloat{\includegraphics[width=0.6\textwidth]{../plots/heapsort_comparativa.png}}
\end{figure}

\begin{figure}[H]
  \centering
  \subfloat{\includegraphics[width=0.6\textwidth]{../plots/floyd_comparativa.png}}
\end{figure}

\begin{figure}[H]
  \centering
  \subfloat{\includegraphics[width=0.6\textwidth]{../plots/hanoi_comparativa.png}}
\end{figure}

\end{document}