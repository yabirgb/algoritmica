\documentclass[spanish]{beamer}

%Language symbols
\usepackage[spanish]{babel}
\selectlanguage{spanish}
\usepackage[utf8]{inputenc}
\usepackage{verbatim}

\usepackage{graphicx}
\usepackage{subfig}

\usepackage{algorithm}
\usepackage{algpseudocode}
\usepackage{pifont}

% Code

\usepackage{listings,textcomp}
\lstset{
  breakatwhitespace,
  language=c++,
  columns=fullflexible,
  keepspaces,
  breaklines,
  tabsize=2, 
  showstringspaces=false,
  extendedchars=true,
  basicstyle=\fontfamily{pcr}\selectfont\scriptsize,
  keywordstyle=\color{orange},
  upquote=true,
  literate={-}{-}1}

%Theme
\usetheme{metropolis}

%Title
\title{Algoritmo Greedy: Problema de los electricistas}
\date{\today}
\author{Yábir G. Benchakhtir}
\institute{Doble Grado en Ingeniería Informática y Matemáticas}
%Document
\begin{document}

\frame{\titlepage}

\begin{frame}\frametitle{Descripción del problema}
  Tenemos un conjunto de tareas $T$ y un electricista. Cada tarea $T[i]$
  tiene asignado un tiempo $t_i$ que, a efectos de implementación, será
  un entero que indicará los segundos necesarios para completarla. El
  dinero que cobra el electricista por cada tarea completada es
  inversamente proporcional al tiempo de espera del cliente. ¿Cuál es la
  distribución temporal de las tareas que garantiza la máxima ganancia?
\end{frame}

\begin{frame}\frametitle{Candidato a algoritmo}

  El tiempo de espera total es el siguiente:

  \begin{align}
      T(P) &= p_0 + (p_0 + p_1) + (p_0+ p_1 + p_2)  \dots \\
       &= np_0 + (n-1)p_1 + \dots p_n \\
       &= \sum_{k=0}^{n-1}(n-k)p_k
  \end{align}

  Y queremos minimizar este tiempo.
  
\end{frame}

\begin{frame}\frametitle{Algoritmo propuesto}

  \begin{algorithm}[H]
    \caption{Asignación de tareas}
    \begin{algorithmic}
      \State tasks = $[t_1, t_2, ... , t_n]$
      \State schedule = $[]$
      \Procedure{distribution}{}
      \State schedule = sort(tasks)
      \State return schedule
      \EndProcedure
    \end{algorithmic}
  \end{algorithm}
  
\end{frame}

\begin{frame}\frametitle{Por qué este algoritmo es el mejor}
  Suponemos $P$ un vector no ordenado ascendentemente cuya solución es
  óptima y $P'$ el mismo vector $P$ sobre el que se ha realizado una
  permutación consistente en hacer que para $ l < m$ $p_l < p_m$. Esto es posible ya que no está totalmente ordenado. 
  \begin{align*}
    T(P') &= (n-l)p_l + (n-m)p_m + \sum_{k=0, k \notin \{l,m\}}^{n-1}(n-k)p_k \\
    T(P) &= (n-m)p_l + (n-l)p_m + \sum_{k=0, k \notin \{l,m\}}^{n-1}(n-k)p_k
  \end{align*}
\end{frame}

\begin{frame}

  Como $T(P') > T(P)$
  
  \begin{align*}
  (n-l)p_l+(n-m)p_m &> (n-m)p_l + (n-l)p_m \\
  (n-l-n+m)p_l &> p_m(n-l+m+n) \\
  (m-l)p_l &> p_m(m-l) \\
  (m-l)(p_l-p_m) &> 0
\end{align*}
\end{frame}

\begin{frame}\frametitle{¿Es nuestro algoritmo greedy?}


  \begin{itemize}
  \item \textbf{Un conjunto de candidatos.}

    En nuestro caso los candidatos son las tareas que tenemos que completar ($P$).
    \pause
  \item \textbf{Una lista de candidatos ya usados.}

    La lista de candidatos usados se resume a la lista de tareas que ya están asignadas.
    \pause
   
  \item \textbf{Función para determinar si nuestra solución es candidata
      a pertenecer al conjunto de soluciones válidas.}

    Nuestras soluciones siempre van a ser válidas aunque hay
    soluciones que son mejores que otras.
  
  \end{itemize}
  
  \end{frame}

  \begin{frame}\frametitle{¿Se puede resolver nuestro problema con un algoritmo greedy?}


  \begin{itemize}
  \item \textbf{Una función de selección que indique el mejor
      candidato.}  En nuestro algoritmo siempre vamos eligiendo la
    tarea que se encuentra en nuestro vector de candidatos en la
    primera posición. Nuestra función de selección es sencillamente

    \[
      \lambda(P) = P[0]
    \]

    \pause

  \item \textbf{Una función objetivo.}

    Nuestra función objetivo es la que asigna a cada solución el
    tiempo de espera total de los clientes y que ya se ha mostrado:

    \[
      T(P) = \sum_{k=0}^{n-1}(n-k)p_k
    \]
    
  \end{itemize}
    
\end{frame}

\begin{frame}\frametitle{Que ocurre si tenemos varios electricistas}
  Nuestro principal problema es como realizar el reparto de tareas a
  cada trabajador.

  \pause

  Una vez repartidas siempre usamos el algoritmo descrito
  anteriormente.
\end{frame}

\begin{frame}\frametitle{Algoritmo}
  \begin{algorithm}[H]
    \caption{Asignación de tareas}
    \begin{algorithmic}
      \State workers = $[w_1, w_2, ... w_k]$
      \State tasks = sorted($[t_1, t_2, ... t_n]$)
      \Procedure{distribution}{}
      \State assignations = $\{\}$
      \State currentWorker = 0

      \For{task in tasks}
      \State assignations[workers[currentWorker]] += task
      \State currentWorker = $(currentWorker + 1) \% k$
      \EndFor
      \State return assignations
      \EndProcedure
    \end{algorithmic}
  \end{algorithm}
\end{frame}

\begin{frame}\frametitle{Algoritmo}
  En esta solución aplicamos una idea semejante a la
  anterior. Partimos también de un vector ordenado y además estamos
  repartiendo las tareas de manera que la media de tiempos de cada
  trabajador tiende a ser la misma.
\end{frame}

\begin{frame}\frametitle{Ejemplo}

  \[
    P = [9,8,3,6,1,7,5]
    w = 3 
  \]
  \pause
  
  $$sorted(P) = [1,3,5,6,7,8,9]$$
  \pause
  \[
    w_1 = [1,6,9] \quad
    w_2 = [3,7] \quad
    w_3 = [5,8]
  \]

  \pause

  \[
    T(w_1) = 1 + 7 + 16 = 24 \quad
    T(w_2) = 3 + 10 = 13 \quad
    T(w_3) = 5 + 13 = 18
  \]
\end{frame}
\end{document}

