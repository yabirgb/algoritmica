\documentclass{article}
\usepackage[left=3cm,right=3cm,top=2.5cm,bottom=2cm]{geometry} % page settings
\usepackage{amsmath} % provides many mathematical environments & tools
\usepackage{amssymb}
\usepackage{amsfonts}
\usepackage[spanish]{babel}

\usepackage{multirow}

\usepackage{algorithm}
\usepackage{algpseudocode}
\usepackage{pifont}

\usepackage[utf8]{inputenc}
\setlength{\parindent}{0mm}

\usepackage[parfill]{parskip}

% Para el código
\usepackage{listings}
\usepackage{xcolor}
\definecolor{gray}{rgb}{0.5,0.5,0.5}
\newcommand{\n}[1]{{\color{gray}#1}}
\lstset{numbers=left,numberstyle=\small\color{gray}}

% Entorno para estilo de ejercicios
\setlength{\parindent}{0pt}

\usepackage{color}   %May be necessary if you want to color links
\usepackage{hyperref}
\hypersetup{
    colorlinks=true, %set true if you want colored links
    linktoc=all,     %set to all if you want both sections and subsections linked
    linkcolor=blue,  %choose some color if you want links to stand out
}

\usepackage{graphicx}
\usepackage{subfig}

\usepackage{listings,textcomp}
\lstset{
  breakatwhitespace,
  language=c++,
  columns=fullflexible,
  keepspaces,
  breaklines,
  tabsize=2, 
  showstringspaces=false,
  extendedchars=true,
  basicstyle=\fontfamily{pcr}\selectfont\scriptsize,
  keywordstyle=\color{orange},
  upquote=true,
  literate={-}{-}1
}

\lstset{numbers=left,xleftmargin=3.5em,framexleftmargin=1em}

\title{Aplicación del algoritmo Greedy - Electricistas \\[5mm]
  \Large Algorítmica\\
  \normalsize Doble Grado en Ingeniería Informática y Matemáticas\\[5cm]
}
\author{Yábir García Benchakhtir \\ yabirgb@correo.ugr.es \\[10cm]}

\date{\today}

\begin{document}

\maketitle

\newpage
\tableofcontents
\newpage
\section{Descripción del problema}

Tenemos un conjunto de tareas $P$ y un electricista. Cada tarea $P[i]$
tiene asignado un tiempo $p_i$ que, a efectos de implementación, será
un entero que indicará los segundos necesarios para completarla. El
dinero que cobra el electricista por cada tarea completada es
inversamente proporcional al tiempo de espera del cliente. ¿Cuál es la
distribución temporal de las tareas que garantiza la máxima ganancia?

\section{Solución del problema}

En este problema maximizar las ganancias es equivalente al problema de
minimizar el tiempo de espera.

La suma de los tiempos de espera de cada cliente es:

\begin{align}
  T(P) &= p_0 + (p_0 + p_1) + (p_0+ p_1 + p_2)  \dots \\
       &= np_0 + (n-1)p_1 + \dots p_n \\
       &= \sum_{k=0}^{n-1}(n-k)p_k
\end{align}
  

Es razonable pensar que el tiempo de espera se va a minimizar si
ordenamos las tareas en orden no decreciente de tiempo. De esta manera
estamos consiguiendo minimazar el tiempo de espera para las primeras
tareas y las que requieren más tiempo no penalizan al resto.

\section{Algoritmo propuesto}

Para solucionar este problema se propone como algoritmo el siguiente:

\begin{algorithm}[H]
\caption{Asignación de tareas}
\begin{algorithmic}
\State tasks = $[t_1, t_2, ... ,t_n]$
\State schedule = $[]$
\Procedure{distribution}{}
\State schedule = sort(tasks)
\State return schedule
\EndProcedure
\end{algorithmic}
\end{algorithm}


\subsection{¿Es nuestro algoritmo greedy?}

Si analizamos un algoritmo greedy vemos que estos constan de tres
partes diferenciadas:

\begin{itemize}
\item Un conjunto de candidatos.
\item Una lista de candidatos ya usados.
\item Un criterio (función) solución que dice cuándo un conjunto de
  candidatos forma una solución (no necesariamente óptima).
\item Criterio para dictar cuando una solución es válida.
\item Una función de selección que indique el mejor candidato.
\item Una función objetivo que asocie a cada solución un valor.
\end{itemize}

En este caso el conjunto de candidatos es el conjunto de todas las
tareas.

La lista de candidatos ya usados es la lista que mantenemos con las
tareas en el orden que las va a realizar el electricista.

En este caso, trabajando con enteros, podemos realizar en un
tiempo $nlog(n)$ ya que comparamos en tiempo constante.

Para nosotros todos los subconjuntos de soluciones son válidas.  La
función que nos permita elegir el mejor resultado cada vez es
sencillamente la función

\[
  \lambda(P) = P[0]
\]

Y la función objetivo simplemente es el cálculo del valor que queremos
minimizar.

\section{Demostración de la eficiencia de este algoritmo}

Supongamos que $P$ es una distribución no ordenada en función de
tiempo de las tareas y además óptima.

$P'$ es una permutación de $P$ donde hemos cambiado las posiciones $l$
y $m$ tal que $l < m$ y $p[l] < p[m]$. Esta permutación es distinta de
la anterior ya que habíamos supuesto que no estaba ordenado en orden
creciente de tiempos. De esta manera:

\begin{align*}
  T(P') &= (n-l)p_l + (n-m)p_m + \sum_{k=0, k \notin \{l,m\}}^{n-1}(n-k)p_k \\
  T(P) &= (n-m)p_l + (n-l)p_m + \sum_{k=0, k \notin \{l,m\}}^{n-1}(n-k)p_k
\end{align*}

Por hipótesis $T(P) < T(P')$ por tanto:

\begin{align*}
  (n-l)p_l+(n-m)p_m &> (n-m)p_l + (n-l)p_m \\
  (n-l-n+m)p_l &> p_m(n-l+m+n) \\
  (m-l)p_l &> p_m(m-l) \\
  (m-l)(p_l-p_m) &> 0
\end{align*}

Lo cual es absurdo pues por hipótesis $p_l < p_m$.

\section{Caso de tener múltiples electricistas}

En el caso de tener múltiples electricistas el problema reside en como
asignar el reparto de tareas a cada electricista de modo que una vez
hecho el reparto vamos a aplicar el algoritmo anteriormente descrito a
las tareas de cada electricista.

Supongamos que tenemos $k$ electricistas. Lo primero que tenemos que
hacer es ordenar el vector de tareas, esto es igual a lo que haciamos
en el caso de tener un solo electricista.

Una vez tenemos el vector ordenado repartimos a los $k$ elecetricistas
las primeras $k$ tareas de manera que tenemos el vector $P[k:]$
(subdivisión de P a partir de la posición k) y volvemos a repartir de
igual forma hasta quedarnos sin tareas.

De este manera mantenemos una media de tiempo similar para cada
electricista y nos estamos garantizando que las tareas se realizan en
el menor tiempo posible una vez que han sido asignadas a un
electricista.

Una implementación con pseudo-código sería la siguiente:

\begin{algorithm}[H]
\caption{Asignación de tareas}
\begin{algorithmic}
\State workers = $[w_1, w_2, ... w_k]$
\State tasks = sorted($[t_1, t_2, ... t_n]$)
\Procedure{distribution}{}
\State assignations = $\{\}$
\State currentWorker = 0

\For{task in tasks}
\State assignations[workers[currentWorker]] += task
\State currentWorker = $(currentWorker + 1) \% k$
\EndFor
\State return assignations
\EndProcedure
\end{algorithmic}
\end{algorithm}

En el caso de que la cantidad de tareas fuese menor que el número de
trabajadores disponibles el anterior algoritmo sigue siendo válido ya
que si consideramos $n$ el número de tareas basta aplicar el algoritmo
con las sección de trabajadores $W[:n]$ (lista de los trabajadores hasta
la posición n).

\end{document}
